% arara: indent: {overwrite: yes, modifylinebreaks: yes}
% arara: pdflatex: {draft: yes, shell : yes}
% arara: bibtex
% arara: pdflatex: {draft: yes, shell : yes}
% arara: pdflatex: {shell : yes}
% arara: clean: { extensions: [ log ], files: [ indent ] }
% arara: clean: { extensions: [ aux, bak, bbl, blg, log, toc, out ]}

\documentclass[12pt,a4paper]{article}


% Usepackages =============================================
% Standard et geometrie -----------------------------------
\usepackage[utf8]{inputenc}
\usepackage[T1]{fontenc}
\usepackage[english]{babel}
\usepackage[margin=2cm]{geometry}

% Maths ---------------------------------------------------
\usepackage{amsmath,amssymb,amsfonts,amsthm}
\usepackage{mathtools,mathrsfs}

%Liens et dessins -----------------------------------------
\usepackage[colorlinks]{hyperref}
\usepackage[capitalize,nameinlink]{cleveref}
\usepackage{tikz}


% Couleurs ================================================
% definecolor ---------------------------------------------
\definecolor{coGB}{HTML}{1F77B4}
\definecolor{coOr}{HTML}{FF7F0E}
\definecolor{coWG}{HTML}{2CA02C}
\definecolor{coAR}{HTML}{D62728}
\definecolor{cMBP}{HTML}{9467BD}
\definecolor{coSM}{HTML}{8C564B}
\definecolor{coPP}{HTML}{E377C2}
\definecolor{coTG}{HTML}{7F7F7F}
\definecolor{coAG}{HTML}{BCBD22}
\definecolor{coBB}{HTML}{17BECF}
% hypersetup ----------------------------------------------
\hypersetup{
    colorlinks = True,
    linkcolor  = coAR,
    citecolor  = coWG,
    urlcolor   = coGB,
}


% Environment =============================================
\theoremstyle{definition}
\newtheorem{definition}{Definition}

\theoremstyle{plain}
\newtheorem{lemma}[definition]{Lemma}
\newtheorem{theorem}[definition]{Theorem}
\newtheorem{corollary}[definition]{Corollary}

\theoremstyle{remark}
\newtheorem{remark}[definition]{Remark}
\newtheorem{example}[definition]{Example}


% Maths ===================================================
% mathbb --------------------------------------------------
\newcommand{\bbA}{\mathbb{A}}
\newcommand{\bbB}{\mathbb{B}}
\newcommand{\bbC}{\mathbb{C}}
\newcommand{\bbD}{\mathbb{D}}
\newcommand{\bbN}{\mathbb{N}}
\newcommand{\bbR}{\mathbb{R}}
\newcommand{\bbS}{\mathbb{S}}
\newcommand{\bbT}{\mathbb{T}}
\newcommand{\bbZ}{\mathbb{Z}}

% mathcal --------------------------------------------------
\newcommand{\calD}{\mathcal{D}}
\newcommand{\calM}{\mathcal{M}}
\newcommand{\calN}{\mathcal{N}}
\newcommand{\calP}{\mathcal{P}}
\newcommand{\calV}{\mathcal{V}}

% mathscr --------------------------------------------------
\newcommand{\scrC}{\mathscr{C}}

% mathrm --------------------------------------------------
\newcommand{\rmL}{\mathrm{L}}
\newcommand{\rmH}{\mathrm{H}}

% Constants -----------------------------------------------
\newcommand{\ex}{\mathsf{e}}
\newcommand{\im}{\mathsf{i}}

% Fontions ------------------------------------------------
\newcommand{\bJ}{\mathop{}\!\mathsf{J}}
\newcommand{\bY}{\mathop{}\!\mathsf{Y}}
\newcommand{\Hu}{\mathop{}\!\mathsf{H}^{(1)}}
\newcommand{\Hd}{\mathop{}\!\mathsf{H}^{(2)}}
\newcommand{\bI}{\mathop{}\!\mathsf{I}}
\newcommand{\bK}{\mathop{}\!\mathsf{K}}

\newcommand{\bj}{\mathop{}\!\mathsf{j}}
\newcommand{\by}{\mathop{}\!\mathsf{y}}
\newcommand{\hu}{\mathop{}\!\mathsf{h}^{(1)}}
\newcommand{\hd}{\mathop{}\!\mathsf{h}^{(2)}}
\newcommand{\bi}{\mathop{}\!\mathsf{i}}
\newcommand{\bk}{\mathop{}\!\mathsf{k}}

\newcommand{\frj}{\mathop{}\!\mathfrak{j}}
\newcommand{\fry}{\mathop{}\!\mathfrak{y}}
\newcommand{\frh}{\mathop{}\!\mathfrak{h}^{(1)}}
\newcommand{\fri}{\mathop{}\!\mathfrak{i}}
\newcommand{\frk}{\mathop{}\!\mathfrak{K}}

% Operateurs ----------------------------------------------
\newcommand{\di}[1]{\mathop{}\!\mathrm{d}#1}
\DeclareMathOperator{\OO}{\mathcal{O}}
\DeclareMathOperator{\oo}{\mathcal{\scriptstyle O}}
\DeclareMathOperator{\Div}{div}
\DeclareMathOperator{\Rot}{\mathbf{curl}}
\newcommand{\Ind}[1]{\mathop{}\!\mathbf{1}_{#1}}
\newcommand{\restr}[2]{\left. #1\right\vert_{#2}}

% delimiters ----------------------------------------------
\newcommand{\plr}[1]{\left(#1\right)}
\newcommand{\clr}[1]{\left[#1\right]}
\newcommand{\blr}[1]{\left\lbrace#1\right\rbrace}
\newcommand{\abs}[1]{\left\lvert#1\right\rvert}
\newcommand{\norm}[1]{\left\lVert#1\right\rVert}

% Vecteurs ------------------------------------------------
\newcommand{\vect}[1]{\boldsymbol{#1}}
\newcommand{\vx}{\boldsymbol{x}}

% Keywords ------------------------------------------------
\newcommand{\eps}{\varepsilon}
\newcommand{\comp}{\mathrm{comp}}
\newcommand{\loc}{\mathrm{loc}}
\newcommand{\inc}{\mathsf{in}}
\newcommand{\sca}{\mathsf{sc}}
\newcommand{\ecav}{\varepsilon_\mathsf{c}}
\newcommand{\mcav}{\mu_\mathsf{c}}


% Titre ===================================================
\title{claudius: analytic computations of scattering \\ \Large \emph{calculs analytiques pour la diffusion}}
\author{Zoïs \textsc{Moitier}}

%==========================================================
\begin{document}

\maketitle

\begin{abstract}
    We solve scattering problem for cases where we have analytical expressions.
    All cavities consider will be invariant by rotation.
    The shape consider are disk, annulus, ball and spherical shell where the permittivity and permeability are radial function.
\end{abstract}

\tableofcontents

%=========================
\section{General settings}
%=========================

We consider the scattering by spherical cavities, disk or annulus in dimension $2$, and ball or spherical shell in dimension $3$.
We call $\abs{\,\cdot\,}_d$ the euclidean norm in dimension $d = 2, 3$.
Disk and ball are denoted by $\bbB_\rho^d = \{\vx \in \bbR^d \mid \abs{\vx}_d < \rho\}$ for $\rho > 0$, and annulus and spherical shell are denoted by $\bbA_{\rho, \sigma}^d = \{\vx \in \bbR^d \mid \rho < \abs{\vx}_d < \sigma\}$ for $0 < \rho < \sigma$.
The cavities will be compose of concentric spherical shell with impenetrable or penetrable core and denoted by $\Omega$.
For a cavity with impenetrable core and $N \ge 0$ layers, we have $\Omega = \bbA_{\rho_1, \rho_2}^d \cup \bbA_{\rho_2, \rho_3}^d \cup \cdots \cup \bbA_{\rho_N, \rho_{N+1}}^d$ with $0 < \rho_1 < \rho_2 < \cdots < \rho_{N+1}$.
For a cavity with penetrable core and $N \ge 0$ layers, we have $\Omega = \bbB_{\rho_1}^d \cup \bbA_{\rho_1, \rho_2}^d \cup \cdots \cup \bbA_{\rho_N, \rho_{N+1}}^d$ with $0 = \rho_0 < \rho_1 < \cdots < \rho_{N+1}$.
The boundary/interfaces $\Gamma$ of the different layers of the cavity $\Omega$ is compose of concentric spheres $\Gamma = \rho_1\bbS^{d-1} \cup \rho_2\bbS^{d-1} \cup \cdots \cup \rho_N\bbS^{d-1}$.

\bigskip

We define the function $\eps \in \rmL^\infty(\bbR^d)$ and $\mu \in \rmL^\infty(\bbR^d)$ as
\[
    \eps(\vx) = \begin{dcases}
        \ecav\plr{\abs{\vx}_d} & \text{if } \vx \in \Omega \\
        1                      & \text{otherwise}
    \end{dcases} \qquad
    \text{and} \qquad
    \mu(\vx) = \begin{dcases}
        \mcav\plr{\abs{\vx}_d} & \text{if } \vx \in \Omega \\
        1                      & \text{otherwise}
    \end{dcases}
\]
where
\[
    \ecav = \sum_{n = 0}^{N} \eps_n \Ind{(\rho_n, \rho_{n+1})} \qquad
    \text{and} \qquad
    \mcav = \sum_{n = 0}^{N} \mu_n \Ind{(\rho_n, \rho_{n+1})}
\]
with $\eps_n, \mu_n \in \scrC^\infty([\rho_n, \rho_{n+1}], \bbR^*)$ for $0 \le n \le N$.
For the impenetrable case the index $n = 0$ is not use.
In some context the function can be view as the permittivity ($\eps$) and the permeability ($\mu$) of a material.

\bigskip

A positive wavenumber is noted $k$.
We only consider a plane wave incident field, we have $u^\inc : \vx \mapsto \exp(\im\, k\, \vect{\nu} \cdot \vx)$ with $\vect{\nu} \in \bbS^{d-1}$ the direction of the plane wave.
In the following, the scattered field is noted $u^\sca$ and the total field is noted $u$.

%=============================
\section{Helmholtz's equation}
%=============================

%----------------------------------
\subsection{The scattering problem}
%----------------------------------

We consider three type of problems, if the cavity has a penetrable core, if the cavity has an impenetrable core with homogeneous Dirichlet condition, and if the cavity has an impenetrable core with homogeneous Neumann condition.
The domain of the operator $u \mapsto -\mu^{-1}\Div(\eps^{-1}\, \nabla u)$ of domain depend on the type of problem:
\begin{align*}
    D^\calD & \coloneqq \{u \in \rmL^2(\bbR^d \setminus \bbB_{\rho_1}) \mid \Div(\eps^{-1}\, \nabla u) \in \rmL^2(\bbR^d \setminus \bbB_{\rho_1}) \text{ and } \restr{u}{\rho_1\bbS^{d-1}} = 0\}, \\
    D^\calN & \coloneqq \{u \in \rmL^2(\bbR^d \setminus \bbB_{\rho_1}) \mid \Div(\eps^{-1}\, \nabla u) \in \rmL^2(\bbR^d \setminus \bbB_{\rho_1})\},                                              \\
    D^\calP & \coloneqq \{u \in \rmL^2(\bbR^d) \mid \Div(\eps^{-1}\, \nabla u) \in \rmL^2(\bbR^d)\}
\end{align*}
and we define the ``locale'' version
\begin{align*}
    D_\loc^{\calD/\calN} & \coloneqq \{u \in \rmL_\loc^2(\bbR^d \setminus \bbB_{\rho_1}) \mid \forall \chi \in \scrC_\comp^\infty(\bbR^d),\ \chi u \in D^{\calD/\calN}\}, \\
    D_\loc^\calP         & \coloneqq \{u \in \rmL_\loc^2(\bbR^d) \mid \forall \chi \in \scrC_\comp^\infty(\bbR^d),\ \chi u \in D^\calP\}.
\end{align*}

\bigskip

We define the following scattering problem: Given a wavenumber $k > 0$ and an incident field $u^\inc$, find the scattering field $u^\sca \in D_\loc^{\calD/\calN/\calP}$ such that the total field $u = u^\inc + u^\sca$ satisfy
\begin{equation}\label{eq:scat_pde_im}
    (\calD / \calN)\begin{dcases}
        -\mu^{-1}\Div\plr{\eps^{-1} \nabla u} - k^2 u = 0                   & \text{in } \Omega \text{ and } \bbR^2 \setminus \overline{\bbB_{\rho_{N+1}}^d} \\[1ex]
        u = 0 \qquad \text{or} \qquad \partial_{\vect{n}}u = 0              & \text{on } \rho_1\bbS^{d-1}                                                    \\[1ex]
        \clr{u} = 0 \ \text{and}\ \clr{\eps^{-1} \partial_{\vect{n}} u} = 0 & \text{across } \rho_2\bbS^{d-1} \cup \cdots \cup \rho_{N+1}\bbS^{d-1}          \\[1ex]
        \text{$u^\sca$ is $k$-outgoing}
    \end{dcases}
\end{equation}
in the Dirichlet ($\calD$) or Neumann ($\calN$) case, or
\begin{equation}\label{eq:scat_pde}
    (\calP)\begin{dcases}
        -\mu^{-1}\Div\plr{\eps^{-1} \nabla u} - k^2 u = 0                   & \text{in } \Omega \text{ and } \bbR^2 \setminus \overline{\Omega}     \\[1ex]
        \clr{u} = 0 \ \text{and}\ \clr{\eps^{-1} \partial_{\vect{n}} u} = 0 & \text{across } \rho_1\bbS^{d-1} \cup \cdots \cup \rho_{N+1}\bbS^{d-1} \\[1ex]
        \text{$u^\sca$ is $k$-outgoing}
    \end{dcases}
\end{equation}
in the penetrable case, where $\vect{n} : \Gamma \to \bbS^{d-1}$ are the outward unit normal and $u^\sca$ is $k$-outgoing mean that, for $|\vx| \ge \rho_N$, we have
\begin{subequations}\label{eq:OWC}
    \begin{align}
        u^\sca(\vx)
         & = \sum_{m \in \bbZ} \beta_m\, \Hu_m(k\, r)\, \ex^{\im\, m\, \theta},
         &                                                                                                           & \beta_m \in \bbC,
         &                                                                                                           & d = 2,                 \\
        u^\sca(\vx)
         & = \sum_{\ell = 0}^{+\infty} \sum_{m = -\ell}^\ell \beta_\ell^m\, \hu_\ell(k\, r)\, Y_\ell^m(\theta,\phi),
         &                                                                                                           & \beta_\ell^m \in \bbC,
         &                                                                                                           & d = 3,
    \end{align}
\end{subequations}
with $(r, \theta) \in \bbR_+ \times \bbR / 2\pi\bbZ$ the polar coordinate,
$(r, \theta, \phi) \in \bbR_+ \times [0, \pi] \times \bbR / 2\pi\bbZ$ the spherical coordinate, $Y_\ell^m$ the spherical harmonic, and $z \mapsto \Hu_m(z)$ (resp.\ $z \mapsto \hu_m(z)$) is the cylindrical (resp.\ spherical) Hankel function of the first kind of order $m$.

%----------------------------
\subsection{The 1d reduction}
%----------------------------

We look for solution of problem of the form
\begin{align*}
    u(\vx)
     & = \sum_{m \in \bbZ} w_m(r)\, \ex^{\im\, m\, \theta}
     &                                                                                     & \text{where } w_m(r) \coloneqq \frac{1}{2\pi} \int_0^{2\pi} u(r,\theta)\, \ex^{-\im\, m\, \theta} \di{\theta}
     &                                                                                     & d = 2                                                                                                         \\
    u(\vx)
     & = \sum_{\ell = 0}^{+\infty} \sum_{m = -\ell}^\ell w_{\ell, m}(r)\, Y_\ell^m(\omega)
     &                                                                                     & \text{where } w_{\ell, m}(r) \coloneqq \int_{\bbS^2} u(r, \omega)\, \overline{Y_\ell^m(\omega)} \di{\omega}
     &                                                                                     & d = 3
\end{align*}
with $(r, \theta) \in \bbR_+ \times \bbR / 2\pi\bbZ$ the polar coordinate,
$(r, \theta, \phi) \in \bbR_+ \times [0, \pi] \times \bbR / 2\pi\bbZ$ the spherical coordinate.
Similarly, we write the same kind of expansion for $u^\inc$ and $u^\sca$.
Using \cref{eq:wave_expan_2,eq:wave_expan_3}, we deduce that, in dimension $2$, we have $w_m^\inc(r) = \bJ_m(k\, r)$ for $m \in \bbZ$ and, in dimension $3$, we have $w_{\ell, 0}^\inc(r) = \im^\ell\, \sqrt{\frac{2\ell+1}{4\pi}}\, \bj_\ell(k\, r)$ for $\ell \in \bbN$ and $w_{\ell, m}^\inc(r) = 0$ for $\ell \in \bbN^*$ and $m \in \{-\ell, \ldots, -1, 1, \ldots, \ell\}$.
To treat dimension $2$ and $3$ in a unified manner, we set $w_p^\inc(r) = c_p\frj_p(k\, r)$ and $\widehat{p}_d \in \bbR_+$ where
\[
    \begin{array}{l@{\quad}l@{\quad}l@{\quad}l@{\qquad}l}
        p = m \in \bbZ,    & \frj_p = \bJ_m,    & c_p = 1,                                      & \widehat{p}_d = p^2,    & \text{for } d = 2; \\
        p = \ell \in \bbZ, & \frj_p = \bj_\ell, & c_p = \im^\ell\, \sqrt{\frac{2\ell+1}{4\pi}}, & \widehat{p}_d = p(p+1), & \text{for } d = 3.
    \end{array}
\]
So \cref{eq:scat_pde_im,eq:scat_pde,eq:OWC} reduce a family of problem index by $p$: Given $k > 0$ find $w_p^\sca \in D^{\calD_p/\calN_p/\calP_p}_\loc$ such that $w_p = w_p^\inc + w_p^\sca$ and
\begin{equation}
    (\calD_p/\calN_p) : \begin{dcases}
        -\frac{\mu^{-1}}{r^{d-1}} \partial_r\plr{\frac{r^{d-1}}{\eps}\, \partial_r w_p} + \dfrac{\widehat{p}_d}{r^2\, \eps\mu} w_p - k^2\, w_p = 0 & \text{in } \bigcup_{n=1}^N (\rho_n, \rho_{n+1}) \text{ and } (\rho_{N+1}, +\infty) \\
        w_p(\rho_1) = 0 \qquad \text{or} \qquad w_p'(\rho_1) = 0                                                                                   & \text{on } \{\rho_1\}                                                              \\
        \clr{w_p} = 0 \quad\text{and}\quad \clr{\eps^{-1}\, w_p'} = 0                                                                              & \text{across } \{\rho_2, \ldots, \rho_{N+1}\}                                      \\
        w_p^\sca(r) \propto \frh_p(k\, r)                                                                                                          & r \ge \rho_{N+1}
    \end{dcases}
\end{equation}
in the Dirichlet ($\calD$) or Neumann ($\calN$) case, or
\begin{equation}
    (\calP_p) : \begin{dcases}
        -\frac{\mu^{-1}}{r^{d-1}} \partial_r\plr{\frac{r^{d-1}}{\eps}\, \partial_r w_p} + \dfrac{\widehat{p}_d}{r^2\, \eps\mu} w_p - k^2\, w_p = 0 & \text{in } \bigcup_{n=0}^N (\rho_n, \rho_{n+1}) \text{ and } (\rho_{N+1}, +\infty) \\
        w_0'(0) = 0                                                                                                                                & \text{on } \{0\}                                                                   \\
        \clr{w_p} = 0 \quad\text{and}\quad \clr{\eps^{-1}\, w_p'} = 0                                                                              & \text{across } \{\rho_1, \ldots, \rho_{N+1}\}                                      \\
        w_p^\sca(r) \propto \frh_p(k\, r)                                                                                                          & r \ge \rho_{N+1}
    \end{dcases}
\end{equation}
in the penetrable case, where $\propto$ meaning ``up to a multiplicative constant'', $\frh_p = \Hu_m$ if $d = 2$, $\frh_p = \hu_\ell$ if $d = 3$ and
\begin{align*}
    \calD^{\calD_p/\calN_p}_\loc & \coloneqq \{w \in \rmL_\loc^2((\rho_1, +\infty)) \mid \forall \chi \in \scrC_\comp^\infty(\bbR),\ \chi w \in D^{\calD_p/\calN_p}\}, \\
    \calD^{\calP_p}_\loc         & \coloneqq \{w \in \rmL_\loc^2(\bbR_+^*) \mid \forall \chi \in \scrC_\comp^\infty(\bbR),\ \chi w \in D^{\calP_p}\}
\end{align*}
with $A_p w =  \partial_r(\eps^{-1}\, r^{d-1} \partial_r w) - \widehat{p}_d\, r^{d-3}\, \eps^{-1}\, w$ and
\begin{align*}
    D^{\calD_p} & \coloneqq \{w \in \rmL^2((\rho_1, +\infty), r^{d-1}\di{r}) \mid A_p w \in \rmL^2((\rho_1, +\infty)) \text{ and } w(\rho_1) = 0\}, \\
    D^{\calN_p} & \coloneqq \{w \in \rmL^2((\rho_1, +\infty), r^{d-1}\di{r}) \mid A_p w \in \rmL^2((\rho_1, +\infty))\},                            \\
    D^{\calP_p} & \coloneqq \{w \in \rmL^2(\bbR_+^*, r^{d-1}\di{r}) \mid A_p w  \in \rmL^2(\bbR_+^*)\}.
\end{align*}
In each layers $n$ or the core we assume that we have a solution basis of the ordinary differential equation
\[
    -\frac{1}{r^{d-1}\, \mu_n} \partial_r\plr{\frac{r^{d-1}}{\eps_n}\, \partial_r w_p} + \dfrac{\widehat{p}_d}{r^2\, \eps_n\, \mu_n} w_p - k^2 w_p = 0
\]
noted $f_{n, p}$ and $g_{n, p}$, we also assume that $f_{n, p}$ is continues at $0$.

\begin{example}
    If $\eps_n$ and $\mu_n$ are non zero constants, we have
    \[
        f_{n, p}(r) = \begin{dcases}
            \frj_p(\sqrt{\eps_n\mu_n}\, k\, r)  & \text{if } \Re(\eps_n\mu_n) > 0 \\
            \fri_p(\sqrt{-\eps_n\mu_n}\, k\, r) & \text{if } \Re(\eps_n\mu_n) < 0
        \end{dcases}
    \]
    and
    \[
        g_{n, p}(r) = \begin{dcases}
            \fry_p(\sqrt{\eps_n\mu_n}\, k\, r)  & \text{if } \Re(\eps_n\mu_n) > 0 \\
            \frk_p(\sqrt{-\eps_n\mu_n}\, k\, r) & \text{if } \Re(\eps_n\mu_n) < 0
        \end{dcases}
    \]
    where $\frj_p, \fri_p, \fry_p, \frk_p$ are $\bJ_m, \bI_m, \bY_m, \bK_m$ for the dimension $2$ and $\bj_\ell, \bi_\ell, \by_\ell, \bk_\ell$ for the dimension $3$.
\end{example}

%----------------------------
\subsection{The solution}
%----------------------------

We write the solution $w_p$ has
\begin{equation}
    w_p(r) = \begin{dcases}
        \alpha_{0, p}\, f_{0, p}(r)                              & \text{if } 0 \le r \le \rho_1                                     \\
        \alpha_{n, p}\, f_{n, p}(r) + \beta_{n, p}\, g_{n, p}(r) & \text{if } 1 \le n \le N \text{ and } \rho_n \le r \le \rho_{n+1} \\
        \beta_{N+1, p}\frh_p(k\, r)                              & \text{if } r \ge \rho_{N+1}
    \end{dcases}
\end{equation}
and use the transmission conditions to compute the coefficients $\alpha_{0, p}$, $\alpha_{1, p}, \beta_{1, p}, \ldots, \alpha_{N, p}, \beta_{N, p}$, and $\beta_{N+1, p}$.
For the Dirichlet case, the coefficient are solution of $M_{N, p}^\calD X_{N, p} = J_{N, p}$ where $M_{N, p}^\calD \in \calM_{2N+1}(\bbC)$ and
\[
    M_{N, p}^\calD = \begin{pmatrix}
        f_{0, p}(\rho_1)  & g_{0, p}(\rho_1)                                                                                                                          \\
        f_{1, p}(\rho_2)  & g_{1, p}(\rho_2)  & -f_{2, p}(\rho_2)  & -g_{2, p}(\rho_2)                                                                                \\
        f_{1, p}'(\rho_2) & g_{1, p}'(\rho_2) & -f_{2, p}'(\rho_2) & -g_{2, p}'(\rho_2)                                                                               \\
        \\
        \\
                          &                   &                    &                    & f_{N, p}(\rho_{N+1})  & g_{N, p}(\rho_{N+1})  & \frh_p(k\, \rho_{N+1})      \\
                          &                   &                    &                    & f_{N, p}'(\rho_{N+1}) & g_{N, p}'(\rho_{N+1}) & -k{\frh_p}'(k\, \rho_{N+1})
    \end{pmatrix}
\]

%===========================
\section{Maxwell's equation}
%===========================

\appendix

%======================
\section{Miscellaneous}
%======================

%-----------------------
\subsection{Coordinates}
%-----------------------

%
\subsubsection{Polar}
%

In dimension two, the Cartesian coordinate $\vx = (x, y) \in \bbR^2$ are describe in term of polar coordinate $(r, \theta) \in \bbR_+ \times \bbR / 2\pi\bbZ$ by
\[
    \begin{dcases}
        x = r \cos(\theta) \\
        y = r \sin(\theta)
    \end{dcases} \qquad
    \text{and} \qquad
    \begin{dcases}
        r = \sqrt{x^2 + y^2} \\
        \theta = \arg(x + \im y)
    \end{dcases}
\]

%
\subsubsection{Spherical}
%

In dimension three, the Cartesian coordinate $\vx = (x, y, z) \in \bbR^3$ are describe in term of spherical coordinate $(r, \theta, \phi) \in \bbR_+ \times [0, \pi] \times \bbR / 2\pi\bbZ$ by
\[
    \begin{dcases}
        x = r \sin(\theta) \cos(\phi) \\
        y = r \sin(\theta) \sin(\phi) \\
        z = r \cos(\theta)
    \end{dcases} \qquad
    \text{and} \qquad
    \begin{dcases}
        r = \sqrt{x^2 + y^2 + z^2}        \\
        \theta = \arccos\plr{\frac{z}{r}} \\
        \phi = \arg(x + \im y)
    \end{dcases}.
\]

%------------------------------
\subsection{Bessel's functions}
%------------------------------

%
\subsubsection{Cylindrical}
%

The cylindrical Bessel's function $\bJ_m$, $\bY_m$, $\Hu_m$, and $\Hd_m$ are define in \cite[Sec.~10.2]{NIST:DLMF} and they are solutions of the ODE
\[
    -\frac{1}{z}\plr{z\, w'}' + \frac{m^2}{z^2}w - w = 0.
\]
The cylindrical modified Bessel's function $\bI_m$ and $\bK_m$ are define in \cite[Sec.~10.25]{NIST:DLMF} and they are solutions of the ODE
\[
    -\frac{1}{z}\plr{z\, w'}' + \frac{m^2}{z^2}w + w = 0.
\]

%
\subsubsection{Spherical}
%

The spherical Bessel's function $\bj_m$, $\by_m$, $\hu_m$, $\hd_m$ ($-$) and $\bi_m$, $\bk_m$ ($+$) are define in \cite[Sec.~10.47]{NIST:DLMF} and they are solutions of the ODE
\[
    -\frac{1}{z^2}\plr{z^2\, w'}' + \frac{\ell(\ell+1)}{z^2}w \mp w = 0
\]
We have the following relation between the cylindrical and spherical Bessel function
\[
    \mathsf{f}_\ell(z) = \sqrt{\frac{\pi}{2z}} \mathsf{F}_{\ell+\frac{1}{2}}(z)
\]
where the couples $(\mathsf{f}, \mathsf{F})$ are $(\bj, \bJ)$, $(\by, \bY)$, $(\hu, \Hu)$, $(\hd, \Hd)$, $(\bi, \bI)$, and $(\bk, \bK)$.

%------------------------------
\subsection{Spherical harmonic}
%------------------------------

For $\ell \in \bbN$ and $m \in \{-\ell, \ldots, 0, \ldots, \ell\}$, the spherical harmonic $Y_\ell^m$ in the spherical coordinate is define by
\[
    Y_\ell^m(\theta, \phi) = \sqrt{\frac{2\ell+1}{4\pi}\, \frac{(\ell-m)!
        }{(\ell+m)!}}\, P_\ell^m(\cos(\theta))\, \ex^{\im\, m\, \phi}
\]
where $P_\ell^m$ are the associated Legendre polynomial define by
\[
    P_\ell^m(x) = \frac{(-1)^m}{2^\ell\, \ell!}\, (1-x^2)^{\frac{m}{2}}\, \partial_x^{\ell+m} (x^2-1)^\ell.
\]

%--------------------------------------------------
\subsection{Spherical wave expansion of plane wave}\label{sec:plane_wave_expan}
%--------------------------------------------------

The Jacobi-Anger expansion \cite[Eq.~10.12.1]{NIST:DLMF} states that, for $z \in \bbC$ and $t \in \bbC^*$,
\[
    \ex^{\frac{z}{2}(t-t^{-1})} = \sum_{m \in \bbZ} \bJ_m(z)\, t^m.
\]
For a 2d plane wave of direction $(0, 1)^\intercal$ and wavenumber $k > 0$, we get
\begin{equation}\label{eq:wave_expan_2}
    \ex^{\im\, k\, y} = \sum_{m \in \bbZ} \bJ_m(k\, r)\, \ex^{\im\, m\, \theta}
    = \bJ_0(k\, r) + \sum_{m = 1}^{+\infty} \bJ_m(k\, r)\, \plr{\ex^{\im\, m\, \theta} + (-1)^m\ex^{-\im\, m\, \theta}}
\end{equation}
where $(x, y)$ are the Cartesian coordinates and $(r, \theta)$ the corresponding polar coordinates.

\bigskip

For a 3d plane wave of direction $\vect{\nu} \in \bbS^2$ and wavenumber $k > 0$, we have the following expansion
\[
    \ex^{\im\, k\, \vect{\nu} \cdot \vx} = 4\pi \sum_{\ell = 0}^{+\infty} \im^\ell\, \bj_\ell(k\, r) \sum_{m = -\ell}^\ell \overline{Y_\ell^m(\vect{\nu})} Y_\ell^m(\omega)
\]
where $(x, y)$ are the Cartesian coordinates and $(r, \theta)$ the corresponding polar coordinates.
For the particular direction $\vect{\nu} = (0, 0, 1)^\intercal$ we obtain
\begin{equation}\label{eq:wave_expan_3}
    \ex^{\im\, k\, z} = \sum_{\ell = 0}^{+\infty} \im^\ell \,\sqrt{4\pi(2\ell+1)}\, \bj_\ell(k\, r)\, Y_\ell^0(\omega).
\end{equation}

\bibliographystyle{alpha}
\bibliography{references}

%##############################################################################
\newpage
%##############################################################################

%============================
\section{Disk cavities with constant optical parameters}
%============================

\paragraph{Solution of the scattering problem.}
In this section, we assume that the cavity is the unit disk $\bbD$ and that $\ecav$ and $\mcav$ are non zero constant in $[0, 1]$.
The solution of problem  for $0 < r < 1$ depend of the sign of $\ecav\mcav$.
We denote by $C_m$ the function
\[
    C_m(z) = \begin{dcases}
        \bJ_m(\sqrt{\ecav\mcav}\, z)  & \text{if } \Re(\ecav\mcav) > 0 \\
        \bI_m(\sqrt{-\ecav\mcav}\, z) & \text{if } \Re(\ecav\mcav) < 0
    \end{dcases}
\]
where $z \mapsto \bI_m(z)$ is the modified Bessel of the first kind.
The solution of problem  is
\begin{equation}
    w_m(r) = \begin{dcases}
        a_m\, C_m(k\, r)                  & \text{if } r \le 1 \\
        b_m\, \Hu_m(k\, r) + \bJ_m(k\, r) & \text{if } r > 1
    \end{dcases}
\end{equation}
where $(a_m, b_m)$ are solution of
\begin{equation}
    \begin{pmatrix*}[r]
        C_m(k) & -\Hu_m(k)\\[1ex]
        \ecav^{-1}\, C_m'(k) & -{\Hu_m}'(k)
    \end{pmatrix*}
    \begin{pmatrix}
        a_m \\[1ex]
        b_m
    \end{pmatrix} =
    \begin{pmatrix}
        \bJ_m(k) \\[1ex]
        \bJ_m'(k)
    \end{pmatrix}.
\end{equation}

\paragraph{Solution of the resonance problem.}
A resonances $\ell$ is a complex satisfying
\begin{equation}
    \ecav^{-1}\, C_m'(\ell) \Hu_m(\ell) - C_m(\ell) {\Hu_m}'(\ell) = 0
\end{equation}
with the associated mode
\begin{equation}
    w_\ell(r) = \begin{dcases}
        C_m(k\, r)                           & \text{if } r \le 1 \\
        \tfrac{C_m(k)}{\Hu_m(k)}\Hu_m(k\, r) & \text{if } r > 1
    \end{dcases}.
\end{equation}

%============================
\section{Annulus cavities}
%============================

\paragraph{Solution of the scattering problem.}
In this section, we assume that the cavity is the annulus $\bbA_\delta = \{\vx \in \bbR^2 \mid \delta < |\vx| < 1\}$ with $0 < \delta < 1$ and that $\ecav$ and $\mcav$ are non zero constant in $[0, 1]$.
The solution of problem  for $\delta < r < 1$ depend of the sign of $\ecav\mcav$ and are denoted by $C_m$ and $D_m$.
The solution of problem  is
\begin{equation}
    w_m(r) = \begin{dcases}
        a_m\, \bJ_m(k\, r)                  & \text{if } 0 \le r < \delta   \\
        b_m\, C_m(k\, r) + c_m\, D_m(k\, r) & \text{if } \delta \le r \le 1 \\
        d_m\, \Hu_m(k\, r) + \bJ_m(k\, r)   & \text{if } r > 1
    \end{dcases}
\end{equation}
where $(a_m, b_m, c_m, d_m)$ are solution of
\begin{equation}
    \begin{pmatrix*}[r]
        -\bJ_m(k\delta) & C_m(k\delta) & D_m(k\delta) & 0\\[1ex]
        -\bJ_m'(k\delta) & \ecav^{-1}\, C_m'(k\delta) & \ecav^{-1}\, D_m'(k\delta) & 0\\[1ex]
        0 & C_m(k) & D_m(k) & -\Hu_m(k)\\[1ex]
        0 & \ecav^{-1}\, C_m'(k) & \ecav^{-1}\, D_m'(k) & -{\Hu_m}'(k)
    \end{pmatrix*}
    \begin{pmatrix}
        a_m \\[1ex]
        b_m \\[1ex]
        c_m \\[1ex]
        d_m
    \end{pmatrix} =
    \begin{pmatrix}
        0        \\[1ex]
        0        \\[1ex]
        \bJ_m(k) \\[1ex]
        \bJ_m'(k)
    \end{pmatrix}.
\end{equation}

\paragraph{Solution of the resonance problem.}
A resonances $\ell$ is a complex satisfying
\begin{equation}
    \ecav^{-1}\, C_m'(\ell) \Hu_m(\ell) - C_m(\ell) {\Hu_m}'(\ell) = 0
\end{equation}
with the associated mode
\begin{equation}
    w_\ell(r) = \begin{dcases}
        C_m(k\, r)                           & \text{if } r \le 1 \\
        \tfrac{C_m(k)}{\Hu_m(k)}\Hu_m(k\, r) & \text{if } r > 1
    \end{dcases}.
\end{equation}

% = = = = = = = = = = = = = =
\subsection{Constant optical parameters}
% = = = = = = = = = = = = = =

\[
    C_m(z) = \begin{dcases}
        \bJ_m(\sqrt{\ecav\mcav}\, z)  & \text{if } \Re(\ecav\mcav) > 0 \\
        \bI_m(\sqrt{-\ecav\mcav}\, z) & \text{if } \Re(\ecav\mcav) < 0
    \end{dcases} \quad
    \text{and} \quad
    D_m(z) = \begin{dcases}
        \bY_m(\sqrt{\ecav\mcav}\, z)  & \text{if } \Re(\ecav\mcav) > 0 \\
        \bK_m(\sqrt{-\ecav\mcav}\, z) & \text{if } \Re(\ecav\mcav) < 0
    \end{dcases}
\]
where $z \mapsto \bK_m(z)$ is the modified Bessel of the second kind.

% = = = = = = = = = = = = = =
\subsection{The flat well version}
% = = = = = = = = = = = = = =

\end{document}
